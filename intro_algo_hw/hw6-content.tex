\problem{15.1-1}
Show how to modify the \textsc{Print-Stations} procedure to print out the stations in increasing order of station number.
\answer

The algorithm is given below:

\begin{algorithm}[H]
\caption{\textsc{Print-Stations-New}$(l, n)$}
\textsc{Print-Stations-Recursive}$(l, l^*, n)$
\end{algorithm}

\begin{algorithm}[H]
\caption{\textsc{Print-Stations-Recursive}$(l, i, j)$}
\If{$j \neq 1$}{
  \textsc{Print-Stations-Recursive}$(l, l_i[j], j - 1)$\\
}
\textbf{Print} ``line '' $i$ ``, station '' $j$
\end{algorithm}
\qed

\problem{15.1-2}
Use the following equations and the substitution method to show that $r_i(j)$, the number of references made to $f_i[j]$ in a recursive algorithm, equals $2^{n-j}$.
\begin{eqnarray*}
r_1(n)&=& r_2(n) = 1,\\
r_1(j) &=& r_2(j) = r_1(j + 1) + r_2(j + 1).
\end{eqnarray*}
\answer
For the case of $j = n$, we have $r_i(j) = 1 (= 2^{n - j})$. Suppose when $j = n_0$, we have $r_i(j) = 2^{n - j}(=2^{n-n_0})$, then for the case of $j = n_0 - 1$, we have:

\begin{eqnarray*}
r_i(j) &=& r_1(j + 1) + r_2(j + 1)\\
&=& r_1(n_0) + r_2(n_0)\\
&=& 2^{n - n_0} + 2^{n - n_0}\\
&=& 2^{n - (n_0 + 1)}.
\end{eqnarray*}

So by mathematical induction we know that $r_i(j) = 2^{n - j}$.
\qed

\problem{15.2-1}
Find and optimal parenthesization of a matrix-chain product whose sequence of dimensions is $\langle5, 10, 3, 12, 5, 50, 6\rangle$.
\answer
\todo
\qed


\problem{15.2-2}
Give a recursive algorithm \textsc{Matrix-Chain-Multiply}$(A, s, i, j)$ that actually performs the optimal matrix-chain multiplication, given the sequence of matrices 
$\langle A_1, A_2, \ldots, A_n\rangle$, the $s$ table computed by \textsc{Matrix-Chain-Order}, and the indices $i$ and $j$. The initial call would be \textsc{Matrix-Chain-Multiply}$(A, s, 1, n)$.
\answer
\todo
\qed


\problem{15.3-3}
Consider a variant of the matrix-chain multiplication problem in which the goal is to parenthesize the sequence of matrices so as to maximize, rather than minimize, the number of
scalar multiplications. Does this problem exhibit optimal substructure?
\answer
\todo
\qed

\problem{15.3-4}
Describe how assembly-line scheduling has overlapping subproblems.
\answer
\todo
\qed

\problem{15.4-5}
Give an $O(n^2)$-time algorithm to find the longest monotonically increasing subsequence of a sequence of $n$ numbers.
\answer
\todo
\qed

\problem{15.5-2}
Determine the cost and structure of an optimal binary search tree for a set of $n=7$ keys with the following probabilities:
\begin{center}
\begin{tabular}{|c|cccccccc|}
\hline
$i$ & 0 & 1 & 2 & 3 & 4 & 5 & 6 & 7\\
\hline
$p_i$ & & 0.04 & 0.06 & 0.08 & 0.02 & 0.10 & 0.12 & 0.14\\
$q_i$ & 0.06 & 0.06 & 0.06 & 0.06 & 0.05 & 0.05 & 0.05 & 0.05\\
\hline
\end{tabular}
\end{center}
\answer
\todo
\qed

\problem{25.1-4}
Show that matrix multiplication defined by \textsc{Extend-Shortest-Paths} is associative.
\answer
\todo
\qed

\problem{25.2-4}
The \textsc{Floyd-Warshall} algorithm requires $\Theta(n^3)$ space, since we need to compute $d_{ij}^{(k)}$ for $i, j, k = 1, 2, \ldots, n$. Show that the following
procedure, which simply drops all the superscripts, is correct, and thus only $\Theta(n^2)$ space is required.

\begin{algorithm}[H]
\caption{\textsc{Floyd-Warshall'}$(W)$}
$n \leftarrow rows[W]$\\
$D \leftarrow W$\\
\For{$k \leftarrow 1$ \KwTo $n$} {
  \For{$i \leftarrow 1$ \KwTo $n$} {
    \For{$j \leftarrow 1$ \KwTo $n$} {
      $d_{ij}\leftarrow \min(d_{ij}, d_{ik} + d_{kj})$
    }
  }
}
\Return $D$
\end{algorithm}
\answer
\todo
\qed

\problem{25.2-6}
How can the output of the \textsc{Floyd-Warshall} algorithm be used to detect the presence of a negative-weight cycle?
\answer
\todo
\qed


