\documentclass{article}
\usepackage{amsmath,amsfonts,amsthm,amssymb}
\usepackage{setspace}
\usepackage{fancyhdr}
\usepackage{lastpage}
\usepackage{extramarks}
\usepackage{chngpage}
\usepackage{soul,color}
\usepackage{graphicx,float,wrapfig}
\usepackage{programs, algorithm2e, pseudocode}
\usepackage{CJK}
\newcommand{\Class}{Some class}
\newcommand{\ClassInstructor}{Some instructor}

% Homework Specific Information. Change it to your own
\newcommand{\Title}{Homework 1}
\newcommand{\DueDate}{Some day}
\newcommand{\StudentName}{Zhang Yang}
\newcommand{\StudentClass}{J53}
\newcommand{\StudentNumber}{2005011295}

% In case you need to adjust margins:
\topmargin=-0.45in      %
\evensidemargin=0in     %
\oddsidemargin=0in      %
\textwidth=6.5in        %
\textheight=9.6in       %
\headsep=0.25in         %

% Setup the header and footer
\pagestyle{fancy}                                                       %
\lhead{\StudentName}                                                 %
\chead{\Title}  %
\rhead{\firstxmark}                                                     %
\lfoot{\lastxmark}                                                      %
\cfoot{}                                                                %
\rfoot{Page\ \thepage\ of\ \protect\pageref{LastPage}}                          %
\renewcommand\headrulewidth{0.4pt}                                      %
\renewcommand\footrulewidth{0.4pt}                                      %

%%%%%%%%%%%%%%%%%%%%%%%%%%%%%%%%%%%%%%%%%%%%%%%%%%%%%%%%%%%%%
% Some tools
\newcommand{\enterProblemHeader}[1]{\nobreak\extramarks{#1}{#1 continued on next page\ldots}\nobreak%
                                    \nobreak\extramarks{#1 (continued)}{#1 continued on next page\ldots}\nobreak}%
\newcommand{\exitProblemHeader}[1]{\nobreak\extramarks{#1 (continued)}{#1 continued on next page\ldots}\nobreak%
                                   \nobreak\extramarks{#1}{}\nobreak}%

\newcommand{\homeworkProblemName}{}%
\newcounter{homeworkProblemCounter}%
\newenvironment{homeworkProblem}[1][Problem \arabic{homeworkProblemCounter}]%
  {\stepcounter{homeworkProblemCounter}%
   \renewcommand{\homeworkProblemName}{#1}%
   \section*{\homeworkProblemName}%
   \enterProblemHeader{\homeworkProblemName}}%
  {\exitProblemHeader{\homeworkProblemName}}%

\newcommand{\homeworkSectionName}{}%
\newlength{\homeworkSectionLabelLength}{}%
\newenvironment{homeworkSection}[1]%
  {% We put this space here to make sure we're not connected to the above.

   \renewcommand{\homeworkSectionName}{#1}%
   \settowidth{\homeworkSectionLabelLength}{\homeworkSectionName}%
   \addtolength{\homeworkSectionLabelLength}{0.25in}%
   \changetext{}{-\homeworkSectionLabelLength}{}{}{}%
   \subsection*{\homeworkSectionName}%
   \enterProblemHeader{\homeworkProblemName\ [\homeworkSectionName]}}%
  {\enterProblemHeader{\homeworkProblemName}%

   % We put the blank space above in order to make sure this margin
   % change doesn't happen too soon.
   \changetext{}{+\homeworkSectionLabelLength}{}{}{}}%


\newcommand{\Answer}{\ \\\textbf{Answer:} }
\newcommand{\Proof}{\ \\\textbf{Proof:} }
\newcommand{\Acknowledgement}[1]{\ \\{\bf Acknowledgement:} #1}
\newcommand{\Slash}{$\backslash$}

%%%%%%%%%%%%%%%%%%%%%%%%%%%%%%%%%%%%%%%%%%%%%%%%%%%%%%%%%%%%%



%%%%%%%%%%%%%%%%%%%%%%%%%%%%%%%%%%%%%%%%%%%%%%%%%%%%%%%%%%%%%
% Make title
\title{\textmd{\bf \Class: \Title}\\{\large Instructed by \textit{\ClassInstructor}}\\\normalsize\vspace{0.1in}\small{Due\ on\ \DueDate}}
\date{}
\author{\textbf{\StudentName}\ \ \StudentClass\ \ \StudentNumber}
%%%%%%%%%%%%%%%%%%%%%%%%%%%%%%%%%%%%%%%%%%%%%%%%%%%%%%%%%%%%%

\begin{document}
\begin{CJK*}{UTF8}{gbsn}

\begin{spacing}{1.1}
\maketitle \thispagestyle{empty}

%%%%%%%%%%%%%%%%%%%%%%%%%%%%%%%%%%%%%%%%%%%%%%%%%%%%%%%%%%%%%
% Begin edit from here


文本文本

\begin{homeworkProblem}[Some Problem]

文字文字

\Answer
Some solution. 这是解答  \\
\Proof Some proof. 这是证明 \\

\begin{programf}
some program code here 巍峨:wq

Use package \{programs\} for this style
\end{programf}


\begin{program}
some other program code here
    Yes, they could be indented
\end{program}

\programsurround \SetProgramCounter{1999}
\begin{programf}
You could get the program code surrounded by \Slash{}programsurround
And use \Slash{}SetProgramCounter\{?\} to set the line number
Use \Slash{}emProgram, \Slash{}ttProgram and \Slash{}rmProgram to set default font
Use \Slash{}ProgramIndent[?width?] to set indent, eg \Slash{}ProgramIndent{1cm}
\end{programf}

\programsurround
\begin{programs}
You could set the size of program, using environments:
\Slash{}program, normal size
\Slash{}programl, large
\Slash{}programL, Large
\Slash{}programs, small
\Slash{}programf, foot note size
\Slash{}programsc, scriptsize
\Slash{}programt, tiny
\end{programs}


\begin{algorithm}[H]
\SetLine \KwData{this text} \KwResult{how to write algorithm with \LaTeX2e } initialization\; \While{not at
end of this document}{ read current\; \eIf{understand}{ go to next section\; current section becomes this
one\; }{ go back to the beginning of current section\; } } \caption{How to write algorithms}
\end{algorithm}

\begin{algorithm}[H]
\SetLine \KwData{this text}
\KwResult{how to write algorithm with \LaTeX2e}
initialization\;
\While{not at end of this document}{re}
\caption{How to write algorithms}
\end{algorithm}

\begin{pseudocode}
[ovalbox]{SquareAndMultiply}{x,b,n}
\COMMENT{ Compute $x^b \pmod{n}$}\\
z\GETS 1\\
\WHILE b > 0 \DO
\BEGIN
z \GETS z^2 \pmod{n} \\
\IF b\mbox{ is odd}
\THEN z \GETS z \cdot x \pmod{n} \\
b \GETS \CALL{ShiftRight}{b}
\END\\
\RETURN{z}
\end{pseudocode}

\end{homeworkProblem}

\Acknowledgement{Thank somebody for something...}

% End edit to here
%%%%%%%%%%%%%%%%%%%%%%%%%%%%%%%%%%%%%%%%%%%%%%%%%%%%%%%%%%%%%

\end{spacing}
\end{CJK*}
\end{document}

%%%%%%%%%%%%%%%%%%%%%%%%%%%%%%%%%%%%%%%%%%%%%%%%%%%%%%%%%%%%%
