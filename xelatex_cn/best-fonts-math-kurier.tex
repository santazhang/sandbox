
\documentclass{article}


\usepackage{amsmath,amssymb}
\usepackage[top=10pt, bottom=10pt, left=10pt, right=10pt, paperwidth=6.4in, paperheight=2.5in]{geometry}

\pagestyle{empty}
\setlength{\parindent}{0cm}


%PACKAGEOPTIONS
\usepackage[math]{kurier}

\begin{document}


  \textbf{Theorem 1 (Residue Theorem).}
  Let $f$ be analytic in the region $G$ except for the isolated singularities $a_1,a_2,\ldots,a_m$. If $\gamma$ is a closed rectifiable curve in $G$ which does not pass through any of the points $a_k$ and if $\gamma\approx 0$ in $G$ then
  \[
  \frac{1}{2\pi i}\int_\gamma f = \sum_{k=1}^m n(\gamma;a_k) \text{Res}(f;a_k).
  \]

  \textbf{Theorem 2 (Maximum Modulus).}
  \emph{Let $G$ be a bounded open set in $\mathbb{C}$ and suppose that $f$ is a continuous function on $G^-$ which is analytic in $G$. Then}
  \[
  \max\{|f(z)|:z\in G^-\}=\max \{|f(z)|:z\in \partial G \}.
  \]
  \vspace*{-1em}

  \newcommand{\abc}{abcdefghijklmnopqrstuvwxyz}
  \newcommand{\ABC}{ABCDEFGHIJKLMNOPQRSTUVWXYZ}
  \newcommand{\alphabeta}{\alpha\beta\gamma\delta\epsilon\varepsilon\zeta\eta\theta\vartheta\iota\kappa\varkappa\lambda\mu\nu\xi o\pi\varpi\rho\varrho\sigma\varsigma\tau\upsilon\phi\varphi\chi\psi\omega}
  \newcommand{\AlphaBeta}{\Gamma\Delta\Theta\Lambda\Xi\Pi\Sigma\Upsilon\Phi\Psi\Omega}

  %\ABC \quad $\ABC$

  %\abc \quad $\abc$ \quad $01234567890$

  %$\AlphaBeta$ \quad $\alphabeta$ \quad $\ell\wp\aleph\infty\propto\emptyset\nabla\partial\mho\imath\jmath\hslash\eth$

  $\mathrm{A} \Lambda \Delta \nabla \mathrm{B C D} \Sigma \mathrm{E F} \Gamma \mathrm{G H I J K L M N O} \Theta \Omega \mho \mathrm{P} \Phi \Pi \Xi \mathrm{Q R S T U V W X Y} \Upsilon \Psi \mathrm{Z} $  $ \quad 1234567890 $

  %$\mathit{A \Lambda \Delta B C D E F \Gamma G H I J K L M N O \Theta \Omega P \Phi \Pi \Xi Q R S T U V W X Y \Upsilon \Psi Z }$

  % don't allow overfull boxes
  {\par \tolerance=0 \emergencystretch=100em $a\alpha b \beta c \partial d \delta e \epsilon \varepsilon f \zeta \xi g \gamma h \hbar \hslash \iota i \imath j \jmath k \kappa \varkappa l \ell \lambda m n \eta \theta \vartheta o \sigma \varsigma \phi \varphi \wp p \rho \varrho q r s t \tau \pi u \mu \nu v \upsilon w \omega \varpi x \chi y \psi z$ \linebreak[3] $\infty \propto \emptyset \varnothing \mathrm{d}\eth \backepsilon$\par}

  %$\mathcal{\ABC} \quad \mathbb{\ABC}$

  %\boldmath $\alpha + b = 27$


\end{document}
